\documentclass[12pt]{article}
\usepackage{fancyhdr}

\renewcommand{\labelenumii}{\arabic{enumi}.\arabic{enumii}}

\pagestyle{empty}
\oddsidemargin 0.0in
\textwidth 6.5in
\topmargin -0.75in
\textheight 9.5in

\begin{document}
\pagestyle{fancy}
\fancyhf{}
\fancyfoot[RO]{\small \tt Cl\'ement Helsens  \space  \space \space \space \space \space \normalsize \textrm{\thepage}}


\begin{center}
{\bf \Large Statement of Research Interest} \\
\end{center}


\vskip 0.8 cm

\noindent
As far as I can remember I have always been interested in understanding what the universe is made of, 
it is then natural that my research interests lie at the frontier of fundamental physics. 
For decades, high energy colliders have been offering some of the most interesting experimental 
opportunities to expand our knowledge of fundamental physics and discover new physics phenomena. 
%I am interested in simultaneously pursuing a greater understanding of the Standard Model (SM), including the Higgs boson, and in studying the validity of theories that pose solutions to the 
%fundamental questions not yet answered by the SM of particle physics such as: what is the exact nature of Electroweak Symmetry
%breaking, how the SM particles acquire mass, how possibly unify gravity with the other fundamental forces and what is the nature of dark matter? 
%In order to answer these questions, 
Over the past five years, I have pursued a diverse research program including measurements of $t\bar{t}$ cross sections, measurement of $t\bar{t}$ charge asymmetry, 
searches for new particles such as new heavy quarks, at the LHC and beyond. The unifying theme of this research effort is searches for new physics in the top quark sector. 
My plans and involvements for the next five years could be summarized as follows:\\

%8TeV ttH xs= 0.13  13TeV: 0.5pb 
\noindent
{\bf Physics with LHC Run-II data}\\
For the restart of the LHC in 2015 I plan to keep a leading role in physics analyses with top quarks in the final state. 
I plan to establish a strong team with a common software infrastructure in the CERN-ATLAS group to contribute to the study of the $t\bar{t}H, H \rightarrow b\bar{b}$ process.
This channel will certainly allow us to confirm the existence of the fermionic decay modes, and will provide the most accurate measurements of the couplings to $b$.
With a common software infrastructure it would be easy to measure the $t\bar{t}b\bar{b}$ production cross-section and search for heavy vector-like quarks whether produced singly or in pairs.
But to make these measurements possible in ATLAS, we first need to ensure that the experiment can sustain the high instantaneous luminosity expected in the next run.
For example, the jet energy resolution will be degraded by the presence of the particles arising from out-of-time pile-up collisions, thus affecting the reconstruction of the 
invariant mass of di-jet systems such as $H \rightarrow b\bar{b}$. I foresee to contribute to maintaining its good performance.\\

%With the first data at 13~TeV I plan to devote a significant fraction of my time to the search for new heavy particles with top-quarks in the final state such as single vector-like-quark production.
%I also plan to establish a strong team with a common software infrastructure in the CERN-ATLAS group to contribute to the study of the $t\bar{t}H, H \rightarrow b\bar{b}$ process, 

%This channel will allow us to confirm the existence of the fermionic decay modes to the $5\sigma$ level, and will provide the most accurate measurements of the couplings to b.
%I find the 
%I plan to attract new fellows and members of the CAT team to join a common top effort, this will allow several analyses to run the same software infrastructure.
%I also plan to contribute to the performance of the detector with the jet calibration.\\


\noindent
{\bf ATLAS upgrade}\\
I plan to continue to work on the ATLAS new small wheel upgrade by taking a more central role in the quality insurance of the detector with the X-ray system I developed, 
participate in the edging tests to be performed at the new gamma-ray facility at CERN, and to analyze the data taken with the prototype installed in the ATLAS cavern.\\

\noindent
{\bf Future Circular Colliders}\\
%I think it is very important to contribute to the design of the Future Circular Colliders as our field needs new big projects for the coming decades. 
I plan to continue my involvement in the FCC project, where I would be interested having more responsibilities in the performance requirements of the calorimeters.
I also foresee to study physics benchmarks such as signatures involving high $p_T$ top quarks and bosons, or double Higgs production 
and their impact on the detector design.

%\noindent
%In conclusion, I have outlined a continuing physics research program that leverages my expertise in the search for new physics involving top-quarks.
%My research involves top physics as test of SM, contributing to the upgrade of detector and participation to FCC studies.
%This research program is conducted at the energy frontier of physics and has the goal of expanding our fundamental knowledge of nature in the coming decades.\\






%\noindent
%{\bf SM limitations}\\
%With the discovery of a new boson with properties very similar to the SM Higgs boson, the SM can be considered almost complete.
%The Higgs mechanism is the source of Electroweak Symmetry Breaking (EWSB) that explains the source of mass for the fundamental particles in general. 
%However, evidence from a broad number of sources indicate that the SM must be part of a larger theory. For instance, the SM does not explain the presence of 
%Dark-Matter (DM) as observed by astrophysics experiments or provide a theoretical framework for understanding gravity.
%The precise measurement of the $t\bar{t}$ production cross section provides an important test of perturbative 
%QCD calculations and a probe for New Physics in the top quark sector, such as non-standard top quark 
%production mechanisms and/or decay modes. 
%For example, the pair production of top partners decaying into a pair of top-quark and missing transverse momentum 
%($E_{\mathrm{T}}^{\mathrm{miss}}$) $T\bar{T} \rightarrow t\bar{t}A_0A_0$ 
%has the same final state as $t\bar{t}$ pair production with a larger amount of $E_{\mathrm{T}}^{\mathrm{miss}}$ from the undetected $A_0$ pair.
%In supersymmetry models with $R-$parity conservation, $T$ is identified with the stop squark and $A_0$ with the lightest supersymmetric particle, the neutralino ($\chi_0$) 
%or the gravitino ($\tilde{G}$). But in general, the $t\bar{t}+E_{\mathrm{T}}^{\mathrm{miss}}$  signature appears in a set of DM motivated models, as well as in other SM extensions.\\
%%These physics opportunities are newly possible due to the high energy and large integrated luminosities collected by the ATLAS experiment at the Large Hadron Collider (LHC) 
%%and will be expanded by the energy and luminosity upgrade underway at the LHC.\\

%\noindent
%{\bf Search for new physics in the top sector}\\
%On the ATLAS experiment at the Large Hadron Collider (LHC) I lead several analysis group to search for new heavy quarks which appear in many extensions of the SM such as Little Higgs or 
%extra-dimensional models. I performed the searches for Vector-Like-Quarks (VLQ) $t'\bar{t'} \rightarrow Ht+X$ and  $t'\bar{t'} \rightarrow Wb+X$ in the 
%semileptonic channel and the combination of these searches. All the procedures and tools I developed  for those analyses are also used by all the four others 
%vector-like quark searches at 8~TeV.
%I am also one of the main contributor of a quite challenging measurement, $t\bar{t}$ charge asymmetry, since the expected asymmetry from New Physics at the LHC 
%would be quite small, in the few percent level, which requires keeping systematic uncertainties below 1\%. This measurement incorporate a number of experimental
%improvements respect to the $\sqrt{s}$ = 7~TeV analyses and should be the most precise one for LHC Run-I. \\


%\noindent
%{\bf Detector and Future accelerators}\\
%While my main research interest continues to be physics analysis related to the top-quark, and more generally direct and indirect search for new physics, I am very
%interested in detector upgrades needed to deal with higher peak luminosities and to fully exploit the physics opportunities provided by the LHC. 
%I plan to use my current expertise and involve a significant fraction of my time and strong motivation in the 
%ATLAS IBL project 
%or any other project that would allow me 
%to improve my knowledge in detector design and construction.
%In the coming five years at the LHC, we will be able to measure Higgs couplings at the level of 10\% precision and will probe BSM physics at multi-TeV scale. 
%Beyond that, a possible increase in the proton-proton collision energy to well above 13~TeV and new $e^+e^-$ colliders operating at various thresholds have been 
%proposed for studying EWSB with higher precision and extend the discovery reach of new massive particles.
%This is why I think it is very important to contribute to the design of the Future Circular Collider as the future of HEP need new big projects.\\

%\noindent
%{\bf Teaching}\\
%I believe that good teaching is the lifeblood of fundamental research and therefore I am very excited about the opportunity to teach. 
%After having mentored several undergraduate and graduate students in the past four years an I think I now have
%a good understanding of the time and other constraints that apply when teaching in a classroom. \\

%\noindent
%{\bf Future}\\
%BNL
%With the restart of the LHC at higher energy it is very important and interesting to quickly analyze the first data and search for direct hints of new physics. 
%With my expertise in VLQ searches and more generally for new physics search in the top-quark sector, I could quickly participate to new physics searches 
%lead by the  BNL group.\\
%HARVARD
%With the restart of the LHC at higher energy it is very important and interesting to quickly analyze the first data and search for direct hints of new physics. 
%With my expertise in VLQ searches and more generally for new physics search in the top-quark sector, I could quickly participate to the SUSY 
%analysis the Harvard group is involved in (2 leptons + $E_{\mathrm{T}}^{\mathrm{miss}}$, 1 lepton + 2 b + $E_{\mathrm{T}}^{\mathrm{miss}}$) 
%or start activities in the SUSY-3rd generation sector like the search for gluino-mediated stop pair production.\\
%UNIGE
%With the restart of the LHC at higher energy it is very important and interesting to quickly analyze the first data and search for direct hints of new physics. 
%I think at 13~TeV the best way to search for heavy new physics will be using boosted technics and I am really interested and motivated to continue my research in this direction.
%With my expertise in VLQ searches and more generally for new physics search in the top-quark sector, I could quickly participate to the 
%boosted objects and flavor-tagging data analyses the Geneva group is involved in.\\
%SLAC
%With the restart of the LHC at higher energy it is very important and interesting to quickly analyze the first data and search for direct hints of new physics.
%I think at 13~TeV the best way to search for heavy new physics will be using boosted technics and I am really interested and motivated to continue my research in this direction.
%With my expertise in VLQ searches and more generally for new physics search in the top-quark sector, I could quickly participate to the analyses lead by the Stanford University group.\\

%\noindent
%{\bf Summary}\\
%In conclusion, I have outlined a continuing physics research program that leverages my expertise in the search for new physics involving top-quarks.
%My research involves top physics as test of SM and contributing to the upgrade of detector.
%This research program is conducted at the energy frontier of physics and has the goal of expanding our fundamental knowledge of nature in the coming decades.





\end{document}
