\documentclass[12pt]{article}
\usepackage{fancyhdr}

\renewcommand{\labelenumii}{\arabic{enumi}.\arabic{enumii}}

\pagestyle{empty}
\oddsidemargin 0.0in
\textwidth 6.5in
\topmargin -0.75in
\textheight 9.5in

\begin{document}
\pagestyle{fancy}
\fancyhf{}
\fancyfoot[RO]{\small \tt Cl\'ement Helsens  \space  \space \space \space \space \space \normalsize \textrm{\thepage}}

\begin{center}
{\bf \LARGE Statement of Research Interest} \\
\end{center}
\vskip 0.5 cm
\noindent
As far as I can remember I have always been interested in understanding what the universe is made of, it is then natural that my research interests lie at the frontier of fundamental physics. For decades, high energy colliders have been offering some of the most interesting experimental opportunities to expand our knowledge of fundamental physics and discover new physics phenomena. Over the past ten years, I have pursued a diverse research program including precision measurements in the top sector, searches for new particles at the LHC and beyond and development of detector concepts at the FCC. This document summarises my ambitious plans and involvements for the next five/ten years, and would be carried out in conjunction with PhD student(s)/ postdoc(s) if they were available.

%%%%%%%%%%%%%%%%%%%%%%%%%%%%%%%%%%%%%%%%%%%%%%%%%%%%%%%%%%%%%%%%%%%%%%%%%%%%%%
\vskip 1 cm
\noindent
{\bf \Large Precision top physics with LHC Run-II/III datasets}
\vskip 0.2 cm
\noindent
Precision measurements in the top-sector already suffer from large uncertainties arising from what we commonly call modelling. Given that we have a large available dataset and the second long shut down to scrutinise it, I plan to dedicate a significant fraction of my time in precision measurements in the top sector. First, I will perform measurements such as fragmentation of b-quarks, underlying events, and jet shapes in $t\bar{t}$ events aiming at reducing the modelling uncertainties. Second, once modelling uncertainties could have been reduced, I will focus on measurements where significant improvements can be made due to this reduction: Charge asymmetry ($A_C$) and spin observables in $t\bar{t}$ events.


%%%%%%%%%%%%%%%%%%%%%%%%%%%%%%%%%%%%
\vskip 0.3 cm
\noindent
\underline{\bf Improving the $t\bar{t}$ modelling}
\vskip 0.2 cm
\noindent
For precision measurement in the top-sector it is important to verify and quantify our understanding of how b-quarks radiate and hadronize in a hadron collider environment as the current treatment in parton shower generators is tuned to LEP results. The LEP measurement was performed in $e^+e^- \rightarrow Z \rightarrow b\bar{b}$ events yielding in a clean, back to back sample of b-quarks in a color singlet $b\bar{b}$ production with no underlying event or colour re-connection to the beam. The LHC environment is very different as we have non-singlet $b\bar{b}$ production that can play a role in the fragmentation. As we have a very large sample of b-jets from top decays, the idea is to obtain a clean sample by selecting $J\slash\psi \rightarrow \mu^+\mu^-$ and construct moments of the $J\slash\psi$ with respect to the closest jet, assumed to be the b-jet. Then by comparing unfolded distributions to predictions from leading Monte-Carlo models and tunes, it is possible to comment to which extent they agree with the fragmentation $x_B$ in $e^+e^-$. Of course new tunes of generators should be made from this innovative measurement.
\vskip 0.2 cm
\noindent
Analyses like top mass are suffering from a large uncertainty from colour re-connection, or more generally, non-perturbative soft QCD uncertainties. The idea is to perform a measurement of the underlying events in $t\bar{t}$ events to get better tunes for generators. The very preliminary results obtained with the Run-II dataset indicates that the pile-up could be a limiting factor, it would be thus interesting to study possibilities of such measurement with low pile-up at the beginning of Run-III.
\vskip 0.2 cm
\noindent
Another interesting measurement that is missing in the top-sector that is particularly important for final state radiation and that would be important for Monte-Carlo tuning is the jet shapes. The idea would be to start with a quick feasibility study before investing on a full Run-II measurement.


%%%%%%%%%%%%%%%%%%%%%%%%%%%%%%%%%%%%
\vskip 0.3 cm
\noindent
\underline{\bf Precision measurements}
\vskip 0.2 cm
\noindent
In proton-proton collisions, the largest production mechanism for $t\bar{t}$ is through the fusion of gluons, process that is charge symmetric, while $A_C$ arises from $q\bar{q}$ initial states. Thus, in order to precisely measure $A_C$ it is important to enhance the $q\bar{q}$ fraction in the selected sample. There are different ways of achieving this, and I plan to investigate the $t\bar{t}$ production in association with an initial state photon as a feasibility study to understand when this will become statistically interesting. In addition, I foresee to continue my involvement in the ongoing Run-II $A_C$ measurement at a similar level as now in order to ensure that my expertise is fully used for a smooth publication.
\vskip 0.2 cm
\noindent
There are a limited number of observables that are sensitive to a different coefficient of the spin density matrix of $t\bar{t}$ production. The Run-I result is limited by systematics and especially signal modelling uncertainties. I think there is a nice opportunity to use the reduction of the modelling uncertainties, together with the large Run-II dataset and sophisticated unfolding techniques to get a very precise measurement of those spin observables.






%%%%%%%%%%%%%%%%%%%%%%%%%%%%%%%%%%%%%%%%%%%%%%%%%%%%%%%%%%%%%%%%%%%%%%%%%%%%%%
\vskip 1 cm
\noindent
{\bf \Large Future (Circular) Colliders}
\vskip 0.2 cm
\noindent
I think it is very important to contribute to the design of future colliders as our field needs new large scale projects for the coming decades. I also consider as the best working model sharing activities between current running experiments and future projects. I plan to continue my significant involvement in future projects, by defining activities that spans over different areas: detector R\&D, sophisticated reconstruction techniques and physics studies.

%%%%%%%%%%%%%%%%%%%%%%%%%%%%%%%%%%%%
\vskip 0.3 cm
\noindent
\underline{\bf Calorimeters}
\vskip 0.2 cm
\noindent
Given my expertise in the FCC software and in the simulation of the calorimeters, I started to adapt the FCC-hh calorimetry (Liquid Argon+Tile) to an FCC-ee like detector. The ultimate goal is to prove that such calorimeter concept, which is a relatively cheap and well known technology, is suitable to meet the performance requirements of FCC-ee. Some modifications to the design, like attaching directly the silicon PMT to the scintillator, trying to collect Cherenkov light in both the liquid argon electro-magnetic calorimeter by adding SiPMT and in the hadronic-calorimeter by exploring different types of scintillators/crystals are considered as a first step. At the same time it would be interesting to start R\&D with cryostat experts to understand how thin it could be in order to reduce the amount of passive material. For the case of an FCC-ee detector, it would also be worth investigating the possibility of having the magnet before the calorimeter to be cost effective. That implies further R\&D as the magnet cryostat would have to mechanicaly support the calorimeter and at the same time be as transparent as possible.


%%%%%%%%%%%%%%%%%%%%%%%%%%%%%%%%%%%%
\vskip 0.3 cm
\noindent
\underline{\bf Particle-Flow}
\vskip 0.2 cm
\noindent
A particle-flow algorithm (PFA) is responsible for combining tracking and calorimeter measurements, and is a necessary ingredient to achieve the required performance of the calorimeter system in any FCC environment. The project would start by implementing the PFA in the less busy environemnt of FCC-ee, by considering a realistic tracker parametric simulation and the full simulation of the calorimeters from the previous item. A further improvement with respect to current PFA, would be the addition of timing information. It would allow, for example, to triangulate between the neutral energy deposit, the hard scatter interaction vertex and pile-up vertices. Thus assigning neutral particles to a vertex becomes easier. Algorithms that compute observables such as or missing energy or object isolation could also be benefit from timing, i.e, by including only particle-flow candidates that are compatible in space-time with the hard interaction vertex. As a next step, it would be very interesting to further develop the PFA in the FCC-hh environment, which is overwhelmed by pile-up.


%%%%%%%%%%%%%%%%%%%%%%%%%%%%%%%%%%%%
\vskip 0.3 cm
\noindent
\underline{\bf Physics}
\vskip 0.2 cm
\noindent
The $b\bar{b}\gamma\gamma$ decay mode of di-Higgs production is the most promising channel despite the small branching ratio of 0.25\%. The main backgrounds for this measurement are single Higgs production, $\gamma\gamma$+jets, and $\gamma$+jets. Because of the small cross-section, this channel will be statistically limited even with an integrated luminosity of 30~ab$^{-1}$ at $\sqrt{s}=100$~TeV.  Reaching high precision in this measurement will rely on excellent photon and jet energy resolution. The effect of pile-up (PU) on the direct photon energy resolution has been studied in the context of the FCC-hh CDR, but no use of timing information was made. A potentially large impact on the $m_{\gamma\gamma}$ resolution can result from neutral low energy deposits overlapping with the prompt photons. Such neutral PU background can be subtracted using timing information. Likewise, the impact of PU on the dijet invariant mass resolution can be potentially very large and has not been studied with such high levels of PU. A large negative effect on the signal-to-background can be due to the isolation variable being contaminated by large levels of PU, resulting in a small signal yield. I propose to perform a study that aims at using the particle flow develop in the context of FCC-ee at the FCC-hh to understand what would be the required timing resolution to obtain the target  $\delta\kappa_\lambda/\kappa_\lambda = 5\%$ precision on the Higgs self-coupling. I expect the usage of timing in the event reconstruction to massively benefit the Higgs self-coupling measurement. This study will represent an important milestone in defining the requirements of the FCC-hh detector. 



%%%%%%%%%%%%%%%%%%%%%%%%%%%%%%%%%%%%%%%%%%%%%%%%%%%%%%%%%%%%%%%%%%%%%%%%%%%%%%
\vskip 1. cm
\noindent
{\bf \Large ATLAS upgrade}
\vskip 0.2 cm
\noindent
While my main research interest continues to be physics analyses related to the top-quark and designing the next generation of collider(s), I am very interested in detector upgrades needed to deal with higher peak luminosities and to fully exploit the physics opportunities provided by the HL-LHC project. I would be very interested to start working on the ATLAS upgrade, by studying topics that are of common interests for HL-LHC and FCC-hh. One of the most appealing and challenging candidate is the pileup mitigation in the tracker. I would also be very interested in studying tracking performance in dense environment, where the tracks are so collimated that they starts to overlap.


%%%%%%%%%%%%%%%%%%%%%%%%%%%%%%%%%%%%
\vskip 1. cm
\noindent
In conclusion, I have outlined a continuing physics research program that leverages my expertise in precision measurements involving top-quarks. My research involves top physics as test of SM, contributing to the upgrade of detector and participation to FCC studies. This research program is conducted at the energy frontier of physics and has the goal of expanding our fundamental knowledge of nature in the coming decades.

\end{document}
